\section{Теоретичні основи БЖД}
\textbf{Безпека }- відсутність загрози кому-небудь або чому-небудь.

\textbf{Небезпека} - це системи, об’єкти, механізми та явища, їх небезпечні параметри, характеристики, властивості, які за певних умов здатні нанести шкоду здоров’ю і життю людини, суспільства, приховують загрозу для середовища існування.

Небезпеки поділяються на:\textit{ потенціальні}, \textit{перманентні} та \textit{тотальні}.

\textbf{Причини виникнення небезпеки} - це збіг обставин, в результаті який виникає небезпека та виникають негативні наслідки такі, як нервові потрясіння, хвороби, травми, інвалідність та іноді смерть.

\textbf{Безпека життєдіяльності}  - це норма життя та роботи людей, параметри навколишнього середовища, при яких з визначеною ймовірністю виключається виникнення небезпеки з негативними наслідками.

\textbf{Ризик} - це кількісна оцінка негативних наслідків за певний час. Це відношення небезпек з негативними наслідками до можливого їх числа за певний проміжок часу. Нульового ризику (абсолютної безпеки) не існує. Для цього існує допустимий ризик. \textbf{Допустимий ризик }- це ризик, який суспільство може забезпечити в даний період і який може бути економічно виправдано з врахуванням соціальних проблем. М\textbf{аксимально допустимий ризик} - це ризик, при якому може постраждати не більше 5\% біогеоценозу.